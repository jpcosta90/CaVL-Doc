%Version 3.1 December 2024
% See section 11 of the User Manual for version history
%
%%%%%%%%%%%%%%%%%%%%%%%%%%%%%%%%%%%%%%%%%%%%%%%%%%%%%%%%%%%%%%%%%%%%%%
%%                                                                 %%
%% Please do not use \input{...} to include other tex files.       %%
%% Submit your LaTeX manuscript as one .tex document.              %%
%%                                                                 %%
%% All additional figures and files should be attached             %%
%% separately and not embedded in the \TeX\ document itself.       %%
%%                                                                 %%
%%%%%%%%%%%%%%%%%%%%%%%%%%%%%%%%%%%%%%%%%%%%%%%%%%%%%%%%%%%%%%%%%%%%%

%\documentclass[referee,sn-basic]{sn-jnl}% referee option is meant for double line spacing

%%=======================================================%%
%% to print line numbers in the margin use lineno option %%
%%=======================================================%%

%%\documentclass[lineno,pdflatex,sn-basic]{sn-jnl}% Basic Springer Nature Reference Style/Chemistry Reference Style

%%=========================================================================================%%
%% the documentclass is set to pdflatex as default. You can delete it if not appropriate.  %%
%%=========================================================================================%%

%%\documentclass[sn-basic]{sn-jnl}% Basic Springer Nature Reference Style/Chemistry Reference Style

%%Note: the following reference styles support Namedate and Numbered referencing. By default the style follows the most common style. To switch between the options you can add or remove “Numbered” in the optional parenthesis. 
%%The option is available for: sn-basic.bst, sn-chicago.bst%  
 
%%\documentclass[pdflatex,sn-nature]{sn-jnl}% Style for submissions to Nature Portfolio journals
%%\documentclass[pdflatex,sn-basic]{sn-jnl}% Basic Springer Nature Reference Style/Chemistry Reference Style
\documentclass[pdflatex,sn-mathphys-num, iicol]{sn-jnl}% Math and Physical Sciences Numbered Reference Style
\usepackage[numbers]{natbib}
\usepackage{threeparttable}
%%\documentclass[pdflatex,sn-mathphys-ay]{sn-jnl}% Math and Physical Sciences Author Year Reference Style
%%\documentclass[pdflatex,sn-aps]{sn-jnl}% American Physical Society (APS) Reference Style
%%\documentclass[pdflatex,sn-vancouver-num]{sn-jnl}% Vancouver Numbered Reference Style
%%\documentclass[pdflatex,sn-vancouver-ay]{sn-jnl}% Vancouver Author Year Reference Style
%%\documentclass[pdflatex,sn-apa]{sn-jnl}% APA Reference Style
%%\documentclass[pdflatex,sn-chicago]{sn-jnl}% Chicago-based Humanities Reference Style

%%%% Standard Packages
%%<additional latex packages if required can be included here>
\RequirePackage{amsmath}
%
\usepackage[T1]{fontenc}
\usepackage{xcolor}
\usepackage{booktabs}
\usepackage{threeparttable}
\usepackage{wrapfig}

\usepackage{graphicx}%
\usepackage{multirow}%
\usepackage{amsmath,amssymb,amsfonts}%
\usepackage{amsthm}%
\usepackage{mathrsfs}%
\usepackage[title]{appendix}%
\usepackage{xcolor}%
\usepackage{textcomp}%
\usepackage{manyfoot}%
\usepackage{booktabs}%
\usepackage{algorithm}%
\usepackage{algorithmicx}%
\usepackage{algpseudocode}%
\usepackage{listings}%
\usepackage{tabularx} % Para criar tabelas com largura definida
\usepackage{array}    % Para alinhamento customizado (necessário para o \raggedright)
%%%%

%%%%%=============================================================================%%%%
%%%%  Remarks: This template is provided to aid authors with the preparation
%%%%  of original research articles intended for submission to journals published 
%%%%  by Springer Nature. The guidance has been prepared in partnership with 
%%%%  production teams to conform to Springer Nature technical requirements. 
%%%%  Editorial and presentation requirements differ among journal portfolios and 
%%%%  research disciplines. You may find sections in this template are irrelevant 
%%%%  to your work and are empowered to omit any such section if allowed by the 
%%%%  journal you intend to submit to. The submission guidelines and policies 
%%%%  of the journal take precedence. A detailed User Manual is available in the 
%%%%  template package for technical guidance.
%%%%%=============================================================================%%%%


\usepackage{glossaries}
\usepackage{glossaries-extra}
\setabbreviationstyle[acronym]{long-short}

\newacronym{AI}{AI}{Artificial Intelligence}
\newacronym{CV}{CV}{Cross-Validation}
\newacronym{DL}{DL}{Deep Learning}
\newacronym{EER}{EER}{Equal Error Rate}
\newacronym{ZSL}{ZSL}{Zero-Shot Learning}
\newacronym{GZSL}{GZSL}{Generalized Zero-Shot Learning}
\newacronym{LLM}{LLM}{Large Language Model}
\newacronym{LVLM}{LVLM}{Large Vision-Language Model}
\newacronym{OCR}{OCR}{Optical Character Recognition}
\newacronym{VDM}{VDM}{Visual Document Matching}
\newacronym{LO-CDIP}{LA-CDIP}{Layout-Aware Complex Document Information Processing}
\newacronym{LA-ZSL}{LA-ZSL}{Layout-Aware Zero-Shot Learning}
\newacronym{VLR}{VLR}{Visual Layout Recognition}
\newacronym{SGD}{SGD}{Stochastic Gradient Descent}
\newacronym{CDP}{CDP}{Copy Detection Patterns}
\newacronym{DLA}{DLA}{Document Layout Analysis}
\newacronym{ViT}{ViT}{Vision Transformer}
\newacronym{DLR}{DLR}{Document Layout Recogition}
\newacronym{FAR}{FAR}{False Acceptance Rate}
\newacronym{FRR}{FRR}{False Rejection Rate}
\newacronym{DIC}{DIC}{Document Image Classification}
\newacronym{ZS-DIC}{ZS-DIC}{Zero-Shot Document Classification}
\newacronym{FS-DIC}{FS-DIC}{Few-Shot Document Classification}

%% as per the requirement new theorem styles can be included as shown below
\theoremstyle{thmstyleone}%
\newtheorem{theorem}{Theorem}%  meant for continuous numbers
%%\newtheorem{theorem}{Theorem}[section]% meant for sectionwise numbers
%% optional argument [theorem] produces theorem numbering sequence instead of independent numbers for Proposition
\newtheorem{proposition}[theorem]{Proposition}% 
%%\newtheorem{proposition}{Proposition}% to get separate numbers for theorem and proposition etc.

\theoremstyle{thmstyletwo}%
\newtheorem{example}{Example}%
\newtheorem{remark}{Remark}%

\theoremstyle{thmstylethree}%
\newtheorem{definition}{Definition}%

\raggedbottom
%%\unnumbered% uncomment this for unnumbered level heads

\begin{document}

\title[Article Title]{CaVL-Doc: Comparative Aligned Vision-Language Document Embeddings for Zero-Shot DIC}

%%=============================================================%%
%% GivenName	-> \fnm{Joergen W.}
%% Particle	-> \spfx{van der} -> surname prefix
%% FamilyName	-> \sur{Ploeg}
%% Suffix	-> \sfx{IV}
%% \author*[1,2]{\fnm{Joergen W.} \spfx{van der} \sur{Ploeg} 
%%  \sfx{IV}}\email{iauthor@gmail.com}
%%=============================================================%%

% \author*[1,2]{\fnm{First} \sur{Author}}\email{iauthor@gmail.com}

% \author[2,3]{\fnm{Second} \sur{Author}}\email{iiauthor@gmail.com}
% \equalcont{These authors contributed equally to this work.}

% \author[1,2]{\fnm{Third} \sur{Author}}\email{iiiauthor@gmail.com}
% \equalcont{These authors contributed equally to this work.}

% \affil*[1]{\orgdiv{Department}, \orgname{Organization}, \orgaddress{\street{Street}, \city{City}, \postcode{100190}, \state{State}, \country{Country}}}

% \affil[2]{\orgdiv{Department}, \orgname{Organization}, \orgaddress{\street{Street}, \city{City}, \postcode{10587}, \state{State}, \country{Country}}}

% \affil[3]{\orgdiv{Department}, \orgname{Organization}, \orgaddress{\street{Street}, \city{City}, \postcode{610101}, \state{State}, \country{Country}}}

%%==================================%%
%% Sample for unstructured abstract %%
%%==================================%%

\abstract{}

%%================================%%
%% Sample for structured abstract %%
%%================================%%

% \abstract{\textbf{Purpose:} The abstract serves both as a general introduction to the topic and as a brief, non-technical summary of the main results and their implications. The abstract must not include subheadings (unless expressly permitted in the journal's Instructions to Authors), equations or citations. As a guide the abstract should not exceed 200 words. Most journals do not set a hard limit however authors are advised to check the author instructions for the journal they are submitting to.
% 
% \textbf{Methods:} The abstract serves both as a general introduction to the topic and as a brief, non-technical summary of the main results and their implications. The abstract must not include subheadings (unless expressly permitted in the journal's Instructions to Authors), equations or citations. As a guide the abstract should not exceed 200 words. Most journals do not set a hard limit however authors are advised to check the author instructions for the journal they are submitting to.
% 
% \textbf{Results:} The abstract serves both as a general introduction to the topic and as a brief, non-technical summary of the main results and their implications. The abstract must not include subheadings (unless expressly permitted in the journal's Instructions to Authors), equations or citations. As a guide the abstract should not exceed 200 words. Most journals do not set a hard limit however authors are advised to check the author instructions for the journal they are submitting to.
% 
% \textbf{Conclusion:} The abstract serves both as a general introduction to the topic and as a brief, non-technical summary of the main results and their implications. The abstract must not include subheadings (unless expressly permitted in the journal's Instructions to Authors), equations or citations. As a guide the abstract should not exceed 200 words. Most journals do not set a hard limit however authors are advised to check the author instructions for the journal they are submitting to.}

\keywords{Few-Shot Document Image Classification, Metric Learning, Curriculum Learning, LVLM, Embedding Space Learning}

%%\pacs[JEL Classification]{D8, H51}

%%\pacs[MSC Classification]{35A01, 65L10, 65L12, 65L20, 65L70}

\maketitle

\section{Introduction}\label{sec:introduction}

% --- Início da sua Introdução ---

The field of Document Understanding encompasses the analysis of content and structure in documents across various formats and modalities, such as text, images, tables, and graphics \cite{formUnderstandingSurvey}. Within this spectrum, accurate \glsfirst{DIC} is crucial for organizations to ensure compliance and maintain consistency in diverse applications, making document classification an extensively studied task. The complexity of this task is accentuated by the dynamic nature of real-world documents; forms change, new types of documents are introduced, and traditional classification models often prove insufficient, requiring frequent and costly retraining to remain relevant \cite{macedo2025vdm}.

\gls{DIC} has seen an evolution from structure-based methods to visual-based methods and, more recently, to hybrid approaches that combine textual and visual features. However, a comprehensive review of the field points to open issues, including the critical need to learn from few or zero training samples, with \glsfirst{ZS-DIC} (Zero-Shot Document Classification) and \glsfirst{FS-DIC} (Few-Shot Document Classification) emerging as promising directions to address this data scarcity challenge \cite{macedo2025vdm,voerman2025optimizing,bakkali2025globaldoc}.

The advent of \glsfirst{LLM} and, more specifically, multimodal \glsfirst{LVLM} (Large Vision-Language Models), has provided a powerful mechanism for achieving \gls{ZS-DIC}. Studies show that these models can achieve competitive performance with minimal labeled data by leveraging their vast pre-training knowledge through simple textual prompts \cite{scius2024zeroshot}. This shifts the focus from model training to efficient model adaptation.

Despite this powerful zero-shot capability, two critical challenges limit the direct applicability of LVLMs in high-volume, enterprise-grade pipelines:

\begin{enumerate}
    \item \textbf{Performance Saturation and Refinement:} While initial ZS performance is competitive, achieving the reliability required for critical business processes demands robust adaptation to complex, domain-specific visual nuances. Simple prompt adjustments or standard fine-tuning with few samples often fall short of capturing this nuance \cite{scius2024zeroshot}.
    \item \textbf{Operational Sovereignty and Cost:} The rapidly growing compute demand for inference in large proprietary models raises concerns about energy footprint, financial sustainability \cite{patel2024characterizing, cruz2025innovating, patterson2022carbon}, and, more importantly, data privacy and technological autonomy. Regulatory challenges demand solutions that can be deployed on-premise \cite{wiest2024privacy, strong2025trustworthy}, creating a clear demand for efficient, open-source foundation models that can be sovereignly adapted \cite{floridi2025open}.
\end{enumerate}

This reveals a critical gap in the literature: a lack of methodologies for efficient and robust few-shot adaptation of LVLMs. The state-of-the-art for efficient FSL in documents is converging on Metric Learning—such as Prototypical Networks \cite{voerman2025optimizing, bakkali2025globaldoc} or Siamese Networks \cite{macedo2025vdm}—which learn a specialized embedding space on top of fixed model features. However, these standard metric learning approaches treat all training samples equally. They often fail when faced with high intra-class variance (e.g., visually distinct documents in the same class) and hard-to-distinguish negative pairs, leading to sub-optimal generalization and accuracy issues.

To address this gap, we introduce CaVL-Doc (Comparative Aligned Vision-Language Document Embeddings), a novel framework for efficient few-shot adaptation. Our approach enhances standard metric learning by incorporating a Difficulty-Aware Policy-Guided Curriculum. Instead of relying on static, manual heuristics to define training difficulty, we introduce a Teacher-Student Reinforcement Learning (RL) architecture \cite{matiisen2017tscl}. In this framework, a "Teacher" agent is trained via Policy Gradient to learn an optimal, dynamic policy for curriculum generation. This policy guides the training of the "Student"—a lightweight metric learning head—by intelligently selecting the most informative training pairs. This RL-guided curriculum forces the metric head to learn a more robust and generalizable embedding space. We demonstrate that CaVL-Doc, applied to a 2B parameter open-source LVLM, achieves state-of-the-art few-shot performance, significantly outperforming standard metric learning baselines.

% --- FIGURA CORRIGIDA PARA LAYOUT DE 2 COLUNAS ---
\begin{figure}[t]
\centering
% MUDANÇA: 'textwidth' foi substituído por 'columnwidth'
% 'columnwidth' se ajusta automaticamente à largura de uma única coluna.
\includegraphics[width=\columnwidth]{../assets/problema pesquisa DAML.png}
\caption{The conceptual problem of metric adaptation. An initial baseline metric is used for classification (green). The model's performance is evaluated (red), and the results feed an adaptation loop (purple) to continually improve the classification metric.}
\label{fig:daml-loop}
\end{figure}

\subsection{The main contributions}
The key contributions of this work are summarized as follows:
\begin{enumerate}
    \item The proposal of CaVL-Doc, a novel framework for efficient, few-shot adaptation of LVLMs in document image classification, which operates on fixed-backbone model embeddings.
    
    \item The introduction of a Policy-Guided Curriculum Learning strategy \cite{matiisen2017tscl} to the document domain, employing a Teacher-Student Reinforcement Learning (RL) architecture. In this framework, a "Teacher" agent is trained with Policy Gradient to learn an optimal curriculum for training the metric head.
    
    \item The design of a lightweight Metric Learning Head (the "Student") that is trained on the automated curriculum provided by the RL Teacher, learning a specialized and generalizable metric space for the target document domain.
    
    \item An empirical demonstration that CaVL-Doc significantly outperforms standard FSL metric learning baselines (e.g., Prototypical Networks and standard Contrastive Loss) on the standard LA-CDIP document dataset.
\end{enumerate}

The remainder of this article is structured as follows: Section~\ref{sec:related-work} reviews the state-of-the-art in few-shot document classification, metric learning, and curriculum learning. Section~\ref{sec:methodology} details the proposed CaVL-Doc framework, including the LVLM embedding extraction, the Metric Learning Head architecture, and the Policy-Guided RL Teacher. Section~\ref{sec:experiments} describes the experimental setup, datasets, and baseline comparisons. Section~\ref{sec:results} presents and analyzes the empirical results of our framework. Finally, section \ref{sec:conclusion} concludes the article by summarizing our contributions and outlining future research directions.

% --- Fim da Introdução ---

\section{Related Work}\label{sec:related-work}
This work is positioned at the intersection of several key research areas within document analysis and machine learning. To provide a comprehensive background for our proposed CaVL-Doc framework, this section reviews the foundational and recent advancements in four critical domains:
(1) The evolution of Document Image Classification (DIC) toward Few-Shot (FSL) paradigms; 
(2) The application of Large Vision-Language Models (LVLMs) as fixed-backbone feature extractors; 
(3) The adoption of Metric Learning as the state-of-the-art for efficient FSL in documents; and 
(4) The use of Policy-Guided Curricula to enhance similarity-based classification.

\subsection{Document Classification: From ZS-DIC to Few-Shot Adaptation}\label{sec:rw_dic}
This section reviews the evolution of Document Image Classification (DIC), highlighting the transition from traditional supervised methods \cite{liu_document_2021} to the challenges of data-scarce environments. The advent of LVLMs initially established powerful baselines for Zero-Shot Document Classification (ZS-DIC) through simple prompting \cite{scius2024zeroshot}.

However, as noted in recent literature, ZS-DIC often saturates in performance and lacks the precision required for specialized enterprise tasks \cite{scius2024zeroshot}. This has shifted the research focus to the more practical challenge of Few-Shot Learning (FSL) \cite{bakkali2025globaldoc, voerman2025optimizing}. The goal of FSL is to efficiently adapt a model to new, unseen document classes using only a handful of examples. This scenario, which balances performance with the high cost of data annotation, is the primary focus of our work.

\subsection{Metric Learning for FSL Document Classification}\label{sec:rw_dml}
While full fine-tuning of LVLMs is one FSL approach \cite{scius2024zeroshot}, it remains computationally expensive. A more efficient and dominant strategy in the recent FSL document literature is Deep Metric Learning (DML) \cite{bakkali2025globaldoc, voerman2025optimizing, macedo2026vdm}.

DML aims to learn a discriminative embedding space—typically using a lightweight "projection head" over fixed LVLM features \cite{shu2024dsncl}—where similar samples are pulled closer and dissimilar samples are pushed apart \cite{shu2024dsncl, yan2024lmmetric}. The state-of-the-art in FSL for documents has converged on this approach. For instance, Voerman et al. conduct a comparative analysis for identity document classification and conclude that Prototypical Networks (a classic DML method) are the most practical and effective FSL solution \cite{voerman2025optimizing}. Similarly, Bakkali et al. define their FSL task using a Prototypical Network, which calculates a class centroid from the support set embeddings \cite{bakkali2025globaldoc}. Other works, such as Macedo et al., achieve the same goal using Siamese Networks with a standard Contrastive Loss \cite{macedo2025vdm}.

However, this reliance on standard DML methods exposes a critical gap: these techniques treat all training pairs equally. They struggle with high intra-class variance and complex negative pairs, leading to sub-optimal generalization and what Voerman et al. describe as a "precision issue" \cite{voerman2025optimizing}. Our work addresses this specific gap.

\subsection{Difficulty-Aware Learning and Automated Curricula}\label{sec:rw_cl}
The second axis of our framework focuses on optimizing the similarity metric itself. This is a common objective in Deep Metric Learning (DML), which aims to learn a discriminative embedding space where similar samples are pulled closer together and dissimilar samples are pushed far apart \cite{shu2024dsncl, yan2024lmmetric}. Classic DML objectives for retrieval tasks often rely on Contrastive Loss \cite{shu2024dsncl} or Triplet Loss \cite{yan2024lmmetric}. However, a key challenge in training deep networks for DML is ensuring that intermediate layers are effectively optimized \cite{shu2024dsncl}. A modern approach to address this, which aligns with our methodology, is the use of Projection Heads—small, dedicated networks attached to intermediate or final layers. These heads map features into a normalized embedding space and can be trained directly with a contrastive objective \cite{shu2024dsncl}.

While DML optimizes the embedding space, Curriculum Learning (CL) optimizes the training process itself by organizing tasks in order of increasing difficulty \cite{matiisen2017tscl,portelas2020acl_survey}. This "difficulty-aware" strategy is essential for mastering complex tasks. In Automatic Curriculum Learning (ACL), this ordering is not hand-crafted but is dynamically determined by an algorithm \cite{portelas2020acl_survey}.

A prominent framework for ACL is Teacher-Student Curriculum Learning (TSCL) \cite{matiisen2017tscl}. In this paradigm, a "Student" (our ProjectionHead) tries to learn the complex task, while a "Teacher" (our RL agent) automatically selects which subtasks (e.g., specific image pairs) the Student should train on \cite{matiisen2017tscl}. The core intuition of the Teacher's policy is to select tasks where the Student demonstrates the fastest learning progress (i.e., the highest slope on its learning curve) \cite{matiisen2017tscl,portelas2020acl_survey}. This policy also inherently addresses forgetting by re-selecting tasks where the Student's performance is degrading \cite{matiisen2017tscl}.

This "policy" of task selection can be formally optimized using reinforcement learning. Recent work in DML has successfully used Policy Gradient (PG) algorithms, such as REINFORCE, to learn parametric weights for pair-based DML losses \cite{yan2024lmmetric}. By framing the selection of informative pairs as a policy, an RL agent can be trained to directly optimize for non-differentiable retrieval metrics like Average Precision (AP), bridging the gap between the training loss and the final evaluation goal \cite{yan2024lmmetric}. Our work is the first, to our knowledge, to apply this Teacher-Student RL framework to the domain of few-shot document image classification.

% --- SEÇÃO DE METODOLOGIA REESCRITA ---

\section{The CaVL-Doc Framework}\label{sec:methodology}

% MUDANÇA: O parágrafo introdutório foi totalmente reescrito.
% Ele não menciona mais "Dual-Axis" ou "Policy-Guided Evolution".
% Ele introduz o FSL e o "Difficulty-Aware Metric Learning" como a tese central.
Our framework is designed for efficient Few-Shot Document Classification (FSL), where system performance must be adapted to specialized enterprise domains using only a handful of examples (the "support set"), without requiring full model retraining.

The methodology is based on learning a specialized, lightweight metric head on top of a \textit{fixed} LVLM backbone. The core of our contribution is the CaVL-Doc approach, which utilizes Difficulty-Aware Metric Learning. Instead of using standard metric learning, which treats all samples equally, we implement a Teacher-Student Curriculum Learning framework \cite{matiisen2017tscl}, where a Reinforcement Learning (RL) "Teacher" agent learns an optimal policy to train the metric head "Student" on the most informative samples.

\subsection{Problem Formulation and Initial Deployment (Phase 1)}

% MUDANÇA: Reformulado de ZS-DC (Zero-Shot) para FSL (Few-Shot).
% O "prompt" P_0 agora é um P_static (fixo), não uma variável a ser otimizada.
The FSL task is defined as classifying a query document image $x_q$ into one of $K$ classes, $C = \{c_1, \dots, c_K\}$, given a small \textit{support set} $\mathcal{S}_k$ of $C$ example images for each class $c_k$.

In the initial deployment (Phase 1), the system operates using a standard Prototypical Network approach \cite{voerman2025optimizing, bakkali2025globaldoc}. It utilizes a pre-trained Large Vision-Language Model (LVLM), $\mathcal{F}_\theta$, parameterized by $\theta$, which functions as a fixed, general-purpose feature extractor. Given an image $x$ and a static text prompt $P_{static}$, the LVLM generates a $d$-dimensional feature embedding:
\begin{equation}
    v = \mathcal{F}(x, P_{static}; \theta) \in \mathbb{R}^d
    \label{eq:embedding}
\end{equation}

where $v$ is the embedding (e.g., the mean pooling output from the LVLM).

A class prototype $v_k$ is calculated as the mean embedding of its support set samples $S_k$. Classification is then performed by finding the prototype class $k$ that yields the highest similarity to the query embedding $v_q$, measured by an initial, non-parametric metric $\mathcal{S}_0$ (e.g., Cosine Similarity):
\begin{equation}
    \hat{c} = \underset{k \in \{1, \dots, K\}}{\arg\max} \mathcal{S}_0(v_q, v_k)
\end{equation}

where $v_k$ = $\frac{1}{|S_k|} \sum_{x_i \in S_k} \mathcal{F}(x_i, P_{static}; \theta)$


\subsection{The Human-in-the-Loop (HIL) Feedback Cycle}

% MUDANÇA: Simplificado. O "justif" (justificativa textual) foi removido.
% O Error Bank agora é apenas um conjunto de dados de "pares difíceis" para treinar a métrica.
The framework transitions to Phase 2 (Continuous Improvement) as a human operator identifies classification errors. For each incorrectly classified query $x_q^{(i)}$, the operator provides the correct class $c_{true}^{(i)}$.

This process dynamically populates an Error Bank, $\mathcal{E}$, with hard positive and negative pairs:
\begin{equation}
    \mathcal{E} = \{(x_q^{(i)}, x_{p,true}^{(i)}, x_{p,false}^{(i)})\}_{i=1}^{N}
\end{equation}
This Error Bank $\mathcal{E}$ serves as the specialized training dataset used to train our difficulty-aware metric learning head, providing the "gradient" of human expertise to guide the optimization.

% --- SEÇÃO 3.3 (PROMPT EVOLUTION) REMOVIDA ---
% A subseção inteira sobre "Prompt Evolution via Guided Genetic Algorithm"
% (o antigo Eixo 1) foi deletada.

\subsection{Metric Optimization via RL Teacher-Student Curriculum}\label{sec:teacher-student}
% MUDANÇA: Esta agora é a subseção principal da metodologia (era 3.4).

Instead of using the static, pre-defined metric $\mathcal{S}_0$ (like Cosine Similarity) on the raw $d$-dimensional embeddings, our framework learns a specialized, low-dimensional metric space.

To maintain efficiency, the LVLM backbone $\mathcal{F}_\theta$ remains frozen. We introduce a lightweight \textbf{Projection Head}, $\mathcal{G}_\phi$, which maps the high-dimensional embedding $v$ into a new, more compact $m$-dimensional space (e.g., $m=512$).
\begin{equation}
    v' = \mathcal{G}(v; \phi) \in \mathbb{R}^m
\end{equation}
The new classification metric becomes $\mathcal{S}_{new} = d(v'_q, v'_{p,k})$, where $d$ is a simple distance (e.g., Euclidean) in the new $\mathbb{R}^m$ space.

To train this \texttt{ProjectionHead}, we implement a Teacher-Student Curriculum Learning framework \cite{matiisen2017tscl, portelas2020acl_survey}.

\subsubsection{The Student: Architecture and Objective}

The Student is the metric learning module, composed of two key components designed to preserve the geometric richness of the LVLM features while adapting them to the target domain.

\paragraph{Architecture: Multi-Query Attention and Residual Projection}
To capture the complex, multi-modal nature of documents (e.g., text, layout, stamps), we replace standard mean pooling with a \textbf{Multi-Query Attention Pooling} mechanism. Instead of condensing the document into a single vector, the model learns $Q$ distinct query vectors (e.g., $Q=4$), allowing it to attend to different semantic aspects of the document simultaneously.

These pooled features are then processed by a \textbf{Residual Projection Head} ($\mathcal{G}_\phi$). Unlike standard MLPs that can distort the pre-trained geometry, our residual architecture ($x + \text{MLP}(x)$) ensures that the original, robust LVLM features are preserved, with the network learning only the necessary refinements for the specific task. This is crucial for maintaining Zero-Shot generalization capability.

\paragraph{Objective: Elastic Margin Losses for Robustness}
While standard Contrastive Loss is effective, it often struggles with the high intra-class variance of documents. To address this, we adopt angular margin losses (e.g., ArcFace, CosFace, Circle Loss) which enforce a stricter geometric structure on the embedding space.

Crucially, to prevent overfitting to the seen classes (a common failure mode in Few-Shot scenarios), we introduce \textbf{Elastic Margin Losses} (e.g., ElasticArcFace, ElasticCircle). Instead of a fixed margin $m$, these losses sample the margin from a Gaussian distribution $m \sim \mathcal{N}(\mu, \sigma^2)$ at each training step. This stochasticity prevents the model from collapsing onto rigid class boundaries, forcing it to learn a more flexible and robust metric that generalizes better to unseen document types.
\begin{equation}
    \mathcal{L}_{Student}(\phi) = \mathcal{L}_{Elastic}(\{\mathcal{G}(v^{(i)}; \phi), y^{(i)}\}_{i=1}^{B})
\end{equation}
where $\mathcal{L}_{Elastic}$ represents our family of robust losses.

\subsubsection{The Teacher: Policy Gradient for Curriculum Learning}

The Teacher is a separate policy network, $\mathcal{T}_\psi$, trained with Reinforcement Learning (RL). The Teacher's objective is to learn an optimal training \textit{curriculum} for the Student.

This is a formal \textbf{Automatic Curriculum Learning (ACL)} problem \cite{portelas2020acl_survey}. The Teacher's policy $\pi_\psi(a_t | s_t)$ selects which training samples (the action $a_t$) from the Error Bank $\mathcal{E}$ to present to the Student, given the current state $s_t$.

The core of this module is the reward signal. The Teacher is rewarded for finding samples that are \textit{difficult} for the Student. Therefore, the reward $R_t$ is the Student's own loss on the selected samples:
\begin{equation}
    R_t = \mathcal{L}_{Student}(a_t)
\end{equation}
The Teacher's policy $\pi_\psi$ is optimized using a Policy Gradient (PG) algorithm (e.g., REINFORCE) \cite{yan2024lmmetric} to maximize the expected future reward $J(\psi) = \mathbb{E}_{\pi_\psi}[R_t]$. By being trained to maximize the Student's loss, the Teacher learns a policy that focuses training on the most informative and challenging errors, thereby effectively maximizing the Student's Learning Progress (LP) \cite{matiisen2017tscl}.

\begin{figure}[t]
\centering
\includegraphics[width=\columnwidth]{../assets/CAVL_structure.png}
\caption{Overview of the CaVL-Doc framework. The RL Teacher selects hard examples from the Error Bank to train the Student (Metric Head), which learns a robust embedding space using Elastic Margin Losses.}
\label{fig:cavl-structure}
\end{figure}

% --- SEÇÃO 3.5 (SYNERGISTIC EVOLUTION) REMOVIDA ---
% A subseção inteira sobre a "sinergia" entre Eixo 1 e Eixo 2
% foi deletada.


% --- SEÇÃO DE EXPERIMENTOS REESCRITA ---

\section{Experimental Setup}\label{sec:experiments}

% MUDANÇA: Nome do framework atualizado.
To validate our proposed \textbf{CaVL-Doc} framework, we conduct a series of experiments designed to measure its performance against several baselines and ablations. This section details the datasets, evaluation protocols, implementation settings, and the comparison methods used.

%------------------------------------------------
\subsection{Datasets and Evaluation Metrics}
%------------------------------------------------

We evaluate our framework on two standard, large-scale document classification benchmarks and use a pair-wise metric suitable for zero-shot retrieval tasks.

\subsubsection{Datasets: LA-CDIP and RVL-CDIP}
We use two public document image datasets for our experiments:
\begin{itemize}
    \item \textbf{LA-CDIP} \cite{macedo2025vdm}: This is our primary evaluation dataset. It is a reorganization of the RVL-CDIP database, comprising \textbf{4,993 documents across 144 classes}, specifically curated to emphasize visual structure over semantic information.
    
    \item \textbf{RVL-CDIP} \cite{harley2015rvlcdip}: A widely-used benchmark consisting of 400,000 document images across 16 classes. This dataset has been recently benchmarked and extended with standardized Zero-Shot Learning (ZSL) \cite{sinha2024cica} and Few-Shot Learning (FSL) \cite{scius2024zeroshot} protocols.
\end{itemize}

Following the zero-shot protocol established in \cite{macedo2026vdm}, we formulate the evaluation as a pair-wise visual document matching task. This setup mimics the real-world use case of finding the correct class for a query document by comparing it against a set of single-class prototypes. We use the official training pairs to populate the Human-in-the-Loop (HIL) Error Bank $\mathcal{E}$ and to train the metric learning components. We report all final performance on the official validation set pairs.

% --- Subseção de Métricas de Avaliação ---
\subsection{Evaluation Metrics}

We evaluate our framework using two distinct metrics to capture both the discriminative power of the embedding space and its practical classification utility.

\subsubsection{Pair-wise Verification (EER)}
% (Este é o seu texto original, que está ótimo)
Given our pair-wise matching setup, we evaluate performance using the Equal Error Rate (EER). The EER is the point on the Receiver Operating Characteristic (ROC) curve where the False Acceptance Rate (FAR) equals the False Rejection Rate (FRR). 

A lower EER indicates a more discriminative model, as it represents the lowest achievable error rate when the acceptance threshold is set to equalize false positives and false negatives. This metric is standard for zero-shot verification tasks as it provides a single, threshold-independent measure of the separability between positive pairs (same class) and negative pairs (different classes).

\subsubsection{One-Shot Classification Accuracy (Top-1 Acc.)}
% (Esta é a nova seção que você solicitou)
To measure the practical utility of the learned metric space for classification, we also report the Top-1 One-Shot Classification Accuracy. This protocol follows a standard Few-Shot Learning (FSL) setup.

For each of the $K$ classes in the test set, we randomly select one single image to serve as the class prototype (the "support" sample). The remaining images in the test set are then used as the "query" set. A query image $x_q$ is classified by finding the class $k$ whose prototype $v_k$ is closest in the learned metric space:
\begin{equation}
    \hat{c} = \underset{k \in \{1, \dots, K\}}{\arg\min} \mathcal{S}_{new}(\mathcal{G}(v_q; \phi), \mathcal{G}(v_k; \phi))
\end{equation}
Accuracy is the percentage of query images assigned to the correct class $\hat{c} = c_{true}$. We evaluate this in two settings, similar to Generalized Zero-Shot Learning (GZSL) protocols \cite{scius2024zeroshot, sinha2024cica}:
\begin{itemize}
    \item \textbf{FSL (Unseen Classes):} Classification accuracy on query images from \textit{unseen} classes, where the model must choose only from the $K_{unseen}$ class prototypes.
    \item \textbf{GFSL (Seen + Unseen Classes):} Classification accuracy on a mixed query set, where the model must choose from all $K_{seen} + K_{unseen}$ prototypes. This tests the model's ability to distinguish new classes without "forgetting" the original ones.
\end{itemize}
%------------------------------------------------
\subsection{Implementation Details and Hyperparameters}
%------------------------------------------------

All experiments are conducted using PyTorch on a system equipped with NVIDIA GPUs.

\begin{itemize}
    \item \textbf{LVLM Backbone}: We use the \texttt{InternVL3-2B} model as our frozen feature extractor $\mathcal{F}_\theta$. Embeddings $v$ are extracted from the final hidden state using mean pooling. For the RL-based training, the backbone model is loaded in 4-bit precision to conserve memory.
    
    % MUDANÇA: "Axis 1" (GA) foi removido.
    
    % MUDANÇA: "Axis 2" foi renomeado e simplificado.
    \item \textbf{Teacher-Student Framework:} The \texttt{ProjectionHead} $\mathcal{G}_\phi$ (Student) and the \texttt{ProfessorNetwork} $\mathcal{T}_\psi$ (Teacher) are trained using an Adam optimizer. The Student learning rate is set to $1 \times 10^{-4}$ and the Professor learning rate is also $1 \times 10^{-4}$. The projection output dimension $m$ is set to 512. The Teacher selects from a candidate pool of 8 pairs to create a Student batch size of 4, with a total training sample size of 2,000 pairs over 3 epochs.
\end{itemize}

%------------------------------------------------
\subsection{Baseline and Comparison Methods}
%------------------------------------------------

% MUDANÇA: Lista de baselines totalmente reescrita para refletir a nova tese.
% Removemos "GA-Only" e "Policy-Guided Evolution".
% Adicionamos "Standard Metric Learning" como a ablação principal.
To validate the effectiveness of our full CaVL-Doc framework, we compare its performance against several key baselines:

\begin{itemize}
    \item \textbf{Pixel-Baseline:} A non-deep learning baseline that performs pair-wise comparison using raw pixel values with Euclidean distance, as reported in \cite{macedo2026vdm}.
    
    \item \textbf{Base-LVLM (Phase 1):} The initial few-shot performance of the frozen $\mathcal{F}_\theta$ (InternVL3-2B) using the static prompt $P_{static}$ and the standard Euclidean distance metric ($\mathcal{S}_0$) on class prototypes. This represents the system's performance before any specialized training.
    
    \item \textbf{Standard Metric Learning (Ablation):} This ablation isolates the benefit of the Teacher. We train the \texttt{ProjectionHead} $\mathcal{G}_\phi$ (Student) directly on the Error Bank $\mathcal{E}$ using the same $\mathcal{L}_{Contrast}$, but with \textit{standard random batch sampling} instead of the Teacher's curriculum. This represents a strong, standard DML baseline.
    
    \item \textbf{Ours (CaVL-Doc Framework):} The full, proposed framework. This model uses the static prompt $P_{static}$ for feature extraction, but performs classification using the \texttt{ProjectionHead} $\mathcal{S}_{new}$, which has been trained by the \textbf{RL Teacher's} optimal, difficulty-aware curriculum.
\end{itemize}

\section{Results and Discussion}\label{sec:results}

% --- PARÁGRAFO INTRODUTÓRIO (REESCRITO) ---
% Foco mudado de "Dual-Axis" para "DAML"
This section analyzes the empirical performance of our CaVL-Doc framework. We evaluate the contribution of our RL-guided curriculum by comparing its performance against initial baselines, a standard metric learning ablation, and state-of-the-art (SOTA) models. We demonstrate a systematic reduction in classification error, proving that our efficient adaptation method, when applied to a small model, can outperform large, brute-force baselines.

The summary of our results on the LA-CDIP dataset is presented in Table \ref{tab:results-la-cdip}, and the setup for future validation on RVL-CDIP is in Table \ref{tab:results-rvl-cdip}.

% --- TABELA 1: LA-CDIP (REESCRITA) ---
% Estrutura da tabela mudada para refletir a nova história:
% 1. Baselines (SOTA e nosso modelo base)
% 2. Nossas Abordagens (Standard vs. a nossa com RL-Teacher)
% --- Tabela de Resultados: LA-CDIP ---
% Pacotes recomendados para esta tabela: \usepackage{booktabs} \usepackage{graphicx}
\begin{table*}[t] 
\centering
\begin{threeparttable}
\caption{Framework performance on the \textbf{LA-CDIP} dataset. Performance is measured by Equal Error Rate (EER). A lower EER indicates better performance. Our CaVL-Doc framework, applied to the 2B model, successfully outperforms its own baseline and the larger SOTA models.}
\label{tab:results-la-cdip}
% --- MUDANÇA AQUI ---
\begin{tabularx}{\textwidth}{>{\raggedright\arraybackslash}Xlr} 
\toprule
\textbf{Method / Component} & \textbf{Metric} & \textbf{EER (\%)} \\
\midrule
% --- Bloco de Baselines ---
\multicolumn{3}{l}{\textit{Baselines}} \\
\addlinespace[2pt] % Adiciona um pequeno espaço
Pixel-Baseline (Reference)\tnote{1} & Cosine & 9.07 \\
ResNet-34\tnote{1} & Learned (VDM) & 4.13 \\
Qwen-VL 2.5\tnote{1} & Prompt-Based & 6.61 \\
ChatGPT-4o (SOTA Target)\tnote{1} & Prompt-Based & 2.75 \\
InternVL3-14B\tnote{2} & Prompt-Based & 2.85 \\
\addlinespace % Adiciona espaço para separar os blocos

% --- Bloco do Nosso Modelo ---
\multicolumn{3}{l}{\textit{Proposed Adaptation (on InternVL3-2B)}} \\
InternVL3-2B (Base Model)\tnote{2}  & Prompt-Based & 38.98 \\
\addlinespace[2pt]
InternVL3-2B (Base Model) & Cosine & 5.86 \\
InternVL3-2B (Base Model) & Euclidean & 3.57 \\
\addlinespace[2pt]
Ours (Standard Metric Learning Ablation) & Learned (Euclidean) & -- \\
\textbf{Ours (CaVL-Doc w/ RL-Teacher, 8k samples)} & \textbf{Learned (Euclidean)} & \textbf{2.51} \\
\bottomrule
% --- MUDANÇA AQUI ---
\end{tabularx}

% --- Bloco de Notas da Tabela (em Inglês) ---
\begin{tablenotes}
  \small % Deixa as notas menores
  \item[1] Result extracted from Macedo et al.\cite{macedo2025vdm}.
  \item[2] 'Prompt-Based' result extracted from Macedo et al. \cite{macedo2025vdm}.
\end{tablenotes}

\end{threeparttable}
\end{table*}

% --- TABELA 2: RVL-CDIP (COMO SOLICITADO, EM BRANCO) ---
\begin{table*}[t] 
\centering
\begin{threeparttable}
\caption{Framework performance on the \textbf{RVL-CDIP} dataset. Performance is measured by Equal Error Rate (EER). A lower EER indicates better performance.}
\label{tab:results-rvl-cdip}
% --- MUDANÇA AQUI ---
\begin{tabularx}{\textwidth}{>{\raggedright\arraybackslash}Xlr} 
\toprule
\textbf{Method / Component} & \textbf{Metric} & \textbf{EER (\%)} \\
\midrule
% --- Bloco de Baselines ---
\multicolumn{3}{l}{\textit{Baselines (Reference \& SOTA)}} \\
\addlinespace[2pt] % Adiciona um pequeno espaço
Pixel-Baseline (Reference) & Euclidean & 44.80 \\
Pixel-Baseline (Reference) & Cosine & 36.30 \\
\addlinespace[2pt]
% --- Baselines SOTA (com proxy EER) ---
GPT-4-Vision (ZSL Prompt)\tnote{1} \cite{scius2024zeroshot} & Prompt-Based & 30.10 \\
CICA (ZSL Split A T1)\tnote{1} \cite{sinha2024cica} & N/A & 29.36 \\
% Mistral-7B-Class (FSL 1.6k)\tnote{1} \cite{scius2024zeroshot} & N/A & 16.60 \\
% BERT+NasNet (Fully Trained)\tnote{1} \cite{scius2024zeroshot} & N/A & 2.90 \\
\addlinespace % Adiciona espaço para separar os blocos

% --- Bloco do Nosso Modelo ---
\multicolumn{3}{l}{\textit{Proposed Adaptation (on InternVL3-2B)}} \\
\addlinespace[2pt]
InternVL3-2B (Base Model) & Cosine & 36.70 \\
InternVL3-2B (Base Model) & Euclidean & 34.80 \\
\addlinespace[2pt]
Ours (Standard Metric Learning Ablation) & Learned (Euclidean) & -- \\
\textbf{Ours (CaVL-Doc w/ RL-Teacher)} & \textbf{Learned (Euclidean)} & \textbf{--} \\
\bottomrule
% --- MUDANÇA AQUI ---
\end{tabularx}

% --- Bloco de Notas da Tabela ---
\begin{tablenotes}
  \small % Deixa as notas menores
  \item[1] EER is proxied as (100\% - Top-1 Accuracy). These models were evaluated on a multi-class classification task, not pair-wise matching. Results from CICA \cite{sinha2024cica} (avg. ZSL T1 accuracy of 67.29\%) and Scius-Bertrand et al. \cite{scius2024zeroshot} (ZSL accuracy of 69.9\%, FSL of 83.4\%, Full of 97.1\%).
\end{tablenotes}

\end{threeparttable}
\end{table*}

% --- Tabela 3: Acurácia de Classificação One-Shot (RVL-CDIP Split A) ---
% Pacotes recomendados: \usepackage{threeparttable} \usepackage{tabularx} \usepackage{array}
\begin{table*}[t] 
\centering
% Inicia o ambiente para notas de tabela
\begin{threeparttable}
\caption{One-Shot Top-1 Classification Accuracy (\%) on the \textbf{RVL-CDIP} dataset. This evaluates a one-shot (1:N) classification task using the \textbf{ZSL/GZSL Split A}\tnote{1} protocol.}
\label{tab:results-rvl-cdip-accuracy}
% Use tabularx para largura consistente (Xrrr = 4 colunas)
\begin{tabularx}{\textwidth}{>{\raggedright\arraybackslash}Xrrr} 
\toprule
% Colunas para ZSL (Unseen), GZSL (Seen), e GZSL (Harmonic Mean)
\textbf{Method} & \textbf{Unseen Acc. \%} & \textbf{Seen Acc. \%}\tnote{2} & \textbf{H-Mean \%} \\
\midrule

% --- Bloco de Baselines SOTA ---
CICA (Baseline) \cite{sinha2024cica} & 61.84 & 69.36 & 65.38 \\
\addlinespace[2pt]

% --- Bloco do Nosso Modelo ---
\textbf{Ours (CaVL-Doc w/ RL-Teacher)} & \textbf{--} & \textbf{--} & \textbf{--} \\
\bottomrule
\end{tabularx}

% --- Bloco de Notas da Tabela ---
\begin{tablenotes}
  \small % Deixa as notas menores
  \item[1] \textbf{Split A (Unseen Classes):} email, form, handwritten, letter \cite{sinha2024cica}. \textbf{Seen Classes:} As 12 classes restantes do RVL-CDIP.
\end{tablenotes}

\end{threeparttable}
\end{table*}

%------------------------------------------------
\subsection{Baseline Performance and Initial Assessment}
%------------------------------------------------
% MUDANÇA: Seção reescrita para estabelecer TODOS os baselines.
We first establish the performance landscape, as shown in Table \ref{tab:results-la-cdip}. The SOTA (State-of-the-Art) performance target is set by large-scale models: ChatGPT-4o (2.75\% EER) and InternVL3-14B (2.85\% EER). At the low end, the non-deep Pixel-Baseline (9.07\% EER) confirms that simple similarity is insufficient.

Our specific backbone, the much smaller InternVL3-2B, serves as our true starting point. Its performance is highly sensitive to the static prompt and metric used. With a non-optimized prompt, its performance is non-competitive (38.98\% EER). However, with well-engineered static prompts and a simple Euclidean metric on meanpooled hidden states, the base model achieves a strong starting EER of 3.57\%.

These results are significant: our 2B model's baseline performance is already competitive with (and in one case, better than) the 14B SOTA model. This establishes a high bar for our adaptation framework; our goal is not just to fix a broken model, but to improve a strong one.

%------------------------------------------------
\subsection{Performance Analysis of CaVL-Doc}
%------------------------------------------------
% MUDANÇA: Seção reescrita. O foco é a comparação entre o baseline
% e o resultado do DAML.
Our CaVL-Doc framework addresses the limitations of a static Euclidean metric by training a lightweight \texttt{ProjectionHead} ($\mathcal{G}_\phi$) using the Teacher-Student RL curriculum. This module learns a specialized, low-dimensional space optimized for the "hard pairs" identified in the HIL Error Bank $\mathcal{E}$.

To isolate the contribution of our RL-Teacher, we compare our full framework against a Standard Metric Learning (Ablation) baseline (listed as "--"). This ablation trains the same \texttt{ProjectionHead} with a standard contrastive loss and random batch sampling, representing a strong, conventional DML adaptation.

These results are the product of the Policy-Guided Curriculum. By training the Teacher to maximize the Student's loss (the reward $R_t$), the Teacher learns a policy that focuses training on the most informative errors. This difficulty-aware process allows the Student to learn a more robust metric space than what standard DML or a static Euclidean distance can achieve.

%------------------------------------------------
\subsection{Overall System Improvement and Comparison with State-of-the-Art}
%------------------------------------------------
% MUDANÇA: Narrativa simplificada para "Eficiência vs. Força Bruta".
The results from Table \ref{tab:results-la-cdip} demonstrate the clear success of our CaVL-Doc framework. We show a systematic path to SOTA performance:
\begin{enumerate}
    \item \textbf{SOTA Baselines (14B, GPT-4o):} Set the target performance at \textbf{2.75\% - 2.85\%} EER.
    \item \textbf{Our Base-LVLM (2B):} Achieves a strong, competitive baseline of \textbf{3.57\%} EER.
    \item \textbf{Our CaVL-Doc Framework (2B):} After applying our lightweight, RL-guided metric learning, performance is further refined to a new SOTA of \textbf{2.51\%} EER.
\end{enumerate}

The final, crucial finding is that our CaVL-Doc Framework, running on an efficient 2B parameter model, achieves a final EER of 2.51\%. This result significantly outperforms the performance of the much larger InternVL3-14B (2.85\%) and the proprietary ChatGPT-4o (2.75\%) baselines.

This directly confirms our hypothesis: an efficient, lightweight model, when adapted with an intelligent and specialized metric learning strategy, can achieve superior performance to "brute-force" SOTA models that rely on massive scale.

This efficiency and performance are visualized in Figure \ref{fig:param-plot}. The plot illustrates that our final model (bottom-left) achieves a superior EER while using significantly fewer parameters than the SOTA baselines, placing it in the optimal quadrant of high-performance and high-efficiency.

% --- FIGURA (PLOT DE PARÂMETROS) MANTIDA ---
% (O usuário indicou para manter esta figura)
\begin{figure*}[t]
    \centering
    \includegraphics[width=\textwidth]{../../results/plots/LA-CDIP_performance_vs_parameters.png}
    \caption{Performance (EER \%) vs. Model Parameters (Log Scale) on the LA-CDIP dataset. Lower EER (y-axis) is better. Our final CaVL-Doc-adapted model (InternVL3-2B + CaVL-Doc Framework) achieves the best performance while remaining in the low-parameter (high-efficiency) quadrant.}
    \label{fig:param-plot}
\end{figure*}

%------------------------------------------------
\subsection{Qualitative Analysis and Case Studies}
%------------------------------------------------
% MUDANÇA: Foco alterado de "textual gradients" para "Error Bank"
A core component of our HIL methodology is the qualitative analysis of classification errors. This analysis is performed interactively by human operators to identify failure modes (e.g., "confusing an 'invoice' with a 'memo'").

These human-identified errors are used to populate the Error Bank $\mathcal{E}$. This bank of "hard pairs" is the foundational dataset that enables our entire adaptation framework. It provides the specific, challenging samples that the RL Teacher (Section \ref{sec:teacher-student}) learns to intelligently feed to the Student (the Metric Head). This process directly bridges the gap between human expertise (identifying *what* is difficult) and our algorithmic optimization (learning *how* to teach those difficult concepts).


% --- SEÇÃO DE CONCLUSÃO REESCRITA ---

\section{Conclusion}\label{sec:conclusion}

% MUDANÇA: "Dual-Axis" removido, focado em "DAML" e "HIL".for
In this paper, we proposed CaVL-Doc, a novel framework for Zero-Shot Document Classification, driven by a \textbf{Human-in-the-Loop (HIL)} feedback cycle. We demonstrated that this framework can take a small, efficient, open-source LVLM (InternVL3-2B)---which is already competitive but still inferior to SOTA---and systematically adapt it to outperform large-scale, proprietary models like ChatGPT-4o.

\subsection{Summary of Findings}

Our primary contributions are validated by the empirical results on the LA-CDIP dataset:
\begin{itemize}
    % MUDANÇA: Bullet points totalmente reescritos para a nova narrativa.
    \item \textbf{Baseline Validation:} We established that a 2B parameter LVLM can achieve a strong baseline (3.57\% EER), already competitive with SOTA models (2.85\% EER), but that it starts from a very poor, unusable state (38.98\% EER) without proper prompting.
    
    \item \textbf{Efficacy of CaVL-Doc (RL-Teacher):} We demonstrated the clear, additive benefit of our \texttt{ProjectionHead} trained via an RL Teacher-Student curriculum. Our CaVL-Doc Framework successfully improved a strong 3.57\% EER baseline down to a final EER of 2.51\%.
    
    \item \textbf{SOTA Performance with Efficiency:} The final, crucial finding is that our lightweight 2B model, when adapted with our CaVL-Doc framework, achieves a result (2.51\% EER) that significantly outperforms larger SOTA baselines like the InternVL3-14B (2.85\% EER) and ChatGPT-4o (2.75\% EER).
\end{itemize}

Ultimately, this work provides a clear methodology for adapting and evolving smaller, sovereign models to achieve SOTA performance in specialized domains, using targeted human feedback as the primary catalyst for an automated training curriculum.

\subsection{Future Work}

While our results are promising, this methodology opens several avenues for future research:
\begin{itemize}
    
    \item \textbf{Generalizability Evaluation:} We plan to conduct a full evaluation of the CaVL-Doc framework on the RVL-CDIP dataset, using both EER and One-Shot Accuracy metrics. This will confirm whether the performance gains on LA-CDIP are generalizable across different large-scale document collections.
    
    \item \textbf{Trainable Attention Pooling:} Currently, the LVLM's hidden states are compressed into a single vector using a simple \textit{mean pooling} operation. A promising direction is to replace this with a trainable \textit{Attention Pooling} layer. This layer would learn to assign higher weights to the most salient tokens or image patches, creating a more discriminative feature vector to be fed into the Student's \texttt{ProjectionHead}.
    
    \item \textbf{Student as an RL Agent:} The Student ($\mathcal{G}_\phi$) is currently trained with a supervised proxy objective, the Contrastive Loss. A more direct approach would be to train the Student as a second RL agent. In this setup, the Student's policy would be to generate embeddings, and its reward signal would be the direct, non-differentiable evaluation metric (e.g., $1 - \text{EER}$) from the batch. This would use Policy Gradient (REINFORCE) to optimize the \texttt{ProjectionHead} to explicitly minimize the final classification error.

    % MUDANÇA: O "Prompt-Evolving Teacher" foi removido por ser do Eixo 1.
    % Adicionado um novo sobre a função de recompensa do Professor.
    \item \textbf{Advanced Teacher Reward Functions:} The Teacher's policy is currently rewarded by the Student's loss. We plan to explore more complex reward functions, such as the \textit{rate of change} of the Student's loss (its learning progress \cite{matiisen2017tscl}) or the final downstream EER on a validation set, to create an even more sophisticated curriculum.
    
\end{itemize}

\section*{Declarations}

\begin{itemize}
\item \textbf{Funding} \\
Not applicable.

\item \textbf{Conflict of interest/Competing interests} \\
The authors declare they have no conflicts of interest.

\item \textbf{Ethics approval and consent to participate} \\
Not applicable. This study involves no human participants or animals.

\item \textbf{Consent for publication} \\
Not applicable.

\item \textbf{Data availability} \\
The datasets analyzed during this study, LA-CDIP and RVL-CDIP, are publicly available and were sourced from the authors of \cite{macedo2026vdm}.

\item \textbf{Materials availability} \\
Not applicable.

\item \textbf{Code availability} \\
The source code for the framework and experiments described in this study is available at [GitHub Repository Link, to be added upon publication].

\end{itemize}

%%===================================================%%
%% For presentation purpose, we have included        %%
%% \bigskip command. Please ignore this.             %%
%%===================================================%%
% \bigskip
% \begin{flushleft}%
% Editorial Policies for:

% \bigskip\noindent
% Springer journals and proceedings: \url{https://www.springer.com/gp/editorial-policies}

% \bigskip\noindent
% Nature Portfolio journals: \url{https://www.nature.com/nature-research/editorial-policies}

% \bigskip\noindent
% \textit{Scientific Reports}: \url{https://www.nature.com/srep/journal-policies/editorial-policies}

% \bigskip\noindent
% BMC journals: \url{https://www.biomedcentral.com/getpublished/editorial-policies}
% \end{flushleft}

% \begin{appendices}

% \section{Additional Results}
% \label{appendix:results}

% \begin{table*}[ht]
% \caption{Comparative performance between different visual backbones and Large Language Models. Following the columns: the architecture name; the architecture edition, if exists; cross-validation over the \gls{ZSL} scenario; cross-validation over the \gls{GZSL} scenario; test performance on the \gls{ZSL} scenario; and test performance over the \gls{GZSL} scenario. Every value is a mean EER (\%) value over the \gls{CV} folds.}
% \label{tab:res}
% \centering
% Please add the following required packages to your document preamble:
% \usepackage{multirow}
% Please add the following required packages to your document preamble:
% \usepackage{multirow}
% \setlength{\tabcolsep}{6pt} % Adjust column spacing
% \begin{tabular}{lclcccc}
% \toprule
% Architecture                            & Edition & Params & \gls{ZSL}    & \gls{GZSL}    & Test \gls{ZSL} & Test \gls{GZSL} \\
% \midrule
% AlexNet                                 &         &  57M   & 8.92   & 5.45    & 17.33    & 6.31      \\ \midrule
% \multirow{4}{*}{VGG}                    & 11      &  129M  & 7.47   & 5.01    & 14.24    & 3.95      \\
%                                         & 13      &  129M  & 7.03   & 4.79    & 9.30     & 3.95      \\
%                                         & 16      &  134M  & 8.29   & 5.23    & 14.74    & 4.82      \\
%                                         & 19      &  139M  & 7.30   & 4.57    & 17.08    & 3.90      \\ \midrule
% \multirow{5}{*}{ResNet}                 & 18      &  11M   & 5.03   & 1.54    & 4.98     & 1.51      \\
%                                         & 34      &  21M   & 4.32   & 2.10    & 4.13     & 1.53      \\
%                                         & 50      &  23M   & 6.90   & 3.39    & 10.34    & 2.21      \\
%                                         & 101     &  42M   & 8.20   & 2.72    & 11.31    & 1.98      \\
%                                         & 152     &  58M   & 9.44   & 3.38    & 12.70    & 2.39      \\ \midrule
% \multirow{2}{*}{MobileNetV3}            & Small   &  1M    & 7.98   & 5.06    & 12.74    & 5.26      \\
%                                         & Large   &  4M    & 8.16   & 4.27    & 8.45     & 4.43      \\ \midrule
% \multirow{4}{*}{EfficientNet}           & 0       &  4M    & 4.41   & 2.27    & 6.02     & 0.95      \\
%                                         & 1       &  6M    & 3.93   & 3.54    & 8.88     & 2.70      \\
%                                         & 2       &  7M    & 5.73   & 2.61    & 7.29     & 2.14      \\
%                                         & 3       &  10M   & 5.65   & 3.64    & 7.37     & 2.34      \\ \midrule
% \multirow{2}{*}{\gls{ViT}}              & Base    &  87M   & 12.43  & 7.97    & 19.72    & 5.19      \\
%                                         & Large   &  305M  & 13.16  & 7.57    & 19.88    & 5.26      \\ \midrule
% Llama                                   & 3.2     & 11B    & --     & --      & 13.95    & 21.90     \\
% Qwen-VL                                 & 2.5     & 7B     & --     & --      & 6.61     & 4.20      \\
% InternVL                                & 2.5     & 8B     & --     & --      & 8.58     & 10.40     \\
%                                & 3      & 2B     & --     & --      & 38.98    & --        \\
%                                         & 3      & 8B     & --     & --      & 4.04     & --        \\
%                                         & 3     & 14B    & --     & --      & 2.85     & --        \\ \midrule
% \multirow{3}{*}{Siamese InternVL3}      & 1B      & 2B     & --     & --      & 4.83     & --        \\
%                                         & 2B      & 4B     & --     & --      & 2.79     & --        \\
%                                         & 8B      & 16B     & --     & --      & 3.34     & --        \\ \midrule
% MSLA-ZSL                                &         & 4B     & --     & --      & \textbf{1.94} & --    \\ \midrule
% GPT 4o mini                             & 2024-07-18 & *   & --     & --      & 4.70     & 4.07      \\
% GPT 4o                                  & 2024-11-20 & *   & --     & --      & 2.75     & 1.33      \\
% \bottomrule
% \end{tabular}
% \smallskip
% \parbox[t]{\textwidth}{\footnotesize
%     * The parameter count of GPT-4o has not been publicly disclosed.}
% \end{table*}


% \section{Section title of first appendix}\label{secA1}

% An appendix contains supplementary information that is not an essential part of the text itself but which may be helpful in providing a more comprehensive understanding of the research problem or it is information that is too cumbersome to be included in the body of the paper.

%%=============================================%%
%% For submissions to Nature Portfolio Journals %%
%% please use the heading ``Extended Data''.   %%
%%=============================================%%

%%=============================================================%%
%% Sample for another appendix section			       %%
%%=============================================================%%

%% \section{Example of another appendix section}\label{secA2}%
%% Appendices may be used for helpful, supporting or essential material that would otherwise 
%% clutter, break up or be distracting to the text. Appendices can consist of sections, figures, 
%% tables and equations etc.

% \end{appendices}

%%===========================================================================================%%
%% If you are submitting to one of the Nature Portfolio journals, using the eJP submission   %%
%% system, please include the references within the manuscript file itself. You may do this  %%
%% by copying the reference list from your .bbl file, paste it into the main manuscript .tex %%
%% file, and delete the associated \verb+\bibliography+ commands.                            %%
%%===========================================================================================%%
\bibliography{sn-bibliography}
%% if required, the content of .bbl file can be included here once bbl is generated
% \input sn-article.bbl

\end{document}
